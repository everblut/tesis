%%%%%%%%%%%%%%%%%%%%%
% Documento maestro %
%%%%%%%%%%%%%%%%%%%%%
\documentclass{fime}

%%%%%%%%%%%%%%%%%%%%%%%%%%%%%%%%%%%%%%%%%%%
% Cargando paquetes y definiendo opciones %
%%%%%%%%%%%%%%%%%%%%%%%%%%%%%%%%%%%%%%%%%%%
% Aquí puedes cargar los paquetes que vas a usar. La clase
% fime ya incluye babel, inputenc, graphicx y los de la AMS.
% Cargar un paquete está a tu libertad (y responsabilidad).
\usepackage{hyperref}
    \hypersetup{breaklinks=true,colorlinks=true,
        linkcolor=black,citecolor=black,urlcolor=black}

%%%%%%%%%%%%%%%%%%%%%
% Definiendo campos %
%%%%%%%%%%%%%%%%%%%%%
\def\titulo{Apoyo a la educación en línea: una API para detección de plagio}
\def\autor{Everardo Medina Palomo}
\def\matricula{1428512}
\def\grado{Ingeniero en Tecnología de Software}
% En caso de que el grado tenga orientación o especialidad llenar el siguiente
% campo dejando un ESPACIO INICIAL, en caso contrario, dejar vacío
\def\orientacion{}
\def\fecha{Abril 2016} % Coloca el mes con mayúscula inicial

\def\asesor{Dra. Satu Elisa Schaeffer}
\def\revisorA{Revisor A}
\def\revisorB{Revisor B}
% En el caso de que tu tesis sea de doctorado activa la variable cambiándola a \doctoradotrue
% y define tus otros dos revisores
\newif\ifdoctorado\doctoradofalse
\def\revisorC{Nombre del revisor C}
\def\revisorD{Nombre del revisor D}
% El visto bueno siempre va
\def\vobo{VoBo}

%%%%%%%%%%%%%%%%%%%%%%%
% Inicia el documento %
%%%%%%%%%%%%%%%%%%%%%%%
\begin{document}

\frontmatter
\pagestyle{main}

%%% Incluye PortillasM si tu tesis es de Maestría
%%% y PortillasD si es de doctorado.
\include{Portadas}
%Agradecimientos

\chapter{Agradecimientos}
\markboth{Agradecimientos}{}

Este espacio es para ofrecer mi profundo agradecimiento a las personas que
causaron una impacto en mi vida,
no solo en lo profesional sino en todos sus aspectos.

Gracias a mis padres Aurora Palomo y Salvador Medina Carrillo,
su amor incondicional ha sido un impulso constante durante mi vida.

Gracias a la Doctora Satu Elisa Schaeffer por mostrar interés en la educación
de calidad, conocerle fue un acontecimiento importante en mi vida estudiantil.

Gracias a mis amigos quienes me ayudaron ver el día a día desde diferentes
perspectivas.


\textit{Los libros me han enseñado
pero las personas me han enseñado aún más.}

%Resumen

\chapter{Resumen}
\markboth{Resumen}{}

{\setlength{\leftskip}{10mm}
\setlength{\parindent}{-10mm}

\autor.

Candidato para obtener el grado de \grado\orientacion.

\uanl.

\fime.

Título del estudio:
\vskip 5mm
\centering
\begin{scshape}
\begin{center}
\begin{tabular}{p{11cm}}
  \centering
  {\large \titulo}
\end{tabular}
\end{center}
\end{scshape}

\noindent Número de páginas: \pageref*{lastpage}.}

%%% Comienza a llenar aquí
\paragraph{Objetivos y método de estudio:}
Con el fin de crear una herramienta que permita mejorar la experiencia del
aprendizaje en línea, se desarrollará una API que aceptará la carga de
documentos bajo la tarea de revisar la similitud entre ellos. De manera general
se busca mejorar la calidad del aprendizaje en línea, contrarrestando el plagio
se podrá impulsar de una manera menos pasiva a los estudiantes a explorar su
creatividad y su capacidad de resolución de problematicas.

\paragraph{Contribuciones y conlusiones:}
Y aquí tus contribuciones y conclusiones. (También es parte del formato).

\bigskip\noindent\begin{tabular}{lc}
\vspace*{-2mm}\hspace*{-2mm}Firma del asesor: & \\
\cline{2-2} & \hspace*{1em}\asesor\hspace*{1em}
\end{tabular}


\tableofcontents
\listoffigures
\listoftables

\mainmatter
\pagestyle{fime}

%%% Haz un documento para cada capítulo
\include{Capitulo}

\appendix
%%% Haz un documento para cada apéndice
\include{Apendice}

\backmatter
\pagestyle{main}

%%% Aquí va la bibliografía, puedes usar el entorno de LaTeX (thebibliography)
%%% o la herramienta BibTeX. En caso de que optes por BibTeX, puedes usar
%%% alguno de los archivos de estilo (mighelbib.bst o mighelnat.bst) incluidos
%%% en el paquete, cuyos diseños armonizan con el diseño de tesis provisto por
%%% fime.cls. Para muestra, basta un botón:
\bibliographystyle{mighelbib}
\bibliography{MiBiblio}

\label{lastpage}
\include{Autobiografia}

\end{document}
